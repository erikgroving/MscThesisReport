\chapter{Conclusion}\label{conclusion}
This thesis addressed the dearth of performance-optimized solutions for conducting online training of neural networks by proposing a novel hardware architecture. The proposed hardware accelerator achieves high performance by exploiting the fine-grain parallelism present at the neuron level during the training process.

After providing a background of neural networks, a software model was implemented to verify the chosen training algorithm for the hardware model. This provided the algorithmic foundation of the hardware model.

We then provided an explanation of the architecture and implementation of the hardware model. By carefully allocating computational kernels among the fully connected layers, the resultant balanced computational scheme allowed for highly parallelized computation. The implemented design was a modular solution that resulted in a flexible design. Furthermore, full-scale testbenches along with a convenient testing process resulted in the successful verification of the design.

The final design for the hardware accelerator was clocked at 50 MHz and satisfies all timing requirements. Furthermore, the implemented design uses low-power compared to GPUs. Experimental results of the hardware accelerator show that the proposed solution achieves a speedup of 17.35 compared to the next best online training model. At the same time, the accelerator is nearly as fast as the GPU model that performed training with a batch size of 50. The experiments also revealed that the bottleneck of the solution was from the MMIO communication between the PS and FPGA rather than the training on the FPGA. The results also showed that 18 bits of fixed-point precision is not enough to  successfully converge to a local optimum during training, rather the training process will degrade in performance after a few epochs as precision error accumulates. 

We then analyzed and discussed the experimental results, highlighting the need for more computational precision while showing the massive potential gains in performance from utilizing fine-grained parallelism. 

Ultimately, this thesis has shown the feasibility of designing a hardware accelerator that uses neuron-level parallelism for the online training of neural networks. Furthermore, there are many potential future optimizations and improvements that would increase both performance and functionality of the proposed hardware accelerator.